\documentclass[10pt, twoside]{article}
% Opciones {{{
\usepackage[pdfa, pdfusetitle, unicode=true]{hyperref}
\usepackage[spanish]{babel}
\usepackage[margin=2.5cm, a4paper]{geometry}
\usepackage{luacode}
\usepackage[shortlabels]{enumitem}
\usepackage{import}
\usepackage{xcolor}
\usepackage{fontspec}
\usepackage[mark, raisemark=0.02\paperheight, marknotags]{gitinfo2}
\usepackage{setspace}
\usepackage{lipsum}

\doublespacing

% Btw I use arch
\setmonofont{InconsolataGo Nerd Font}

\newcommand{\btw}{{\color{arch}\texttt{}} }
\newcommand{\git}{{\color{git}\texttt{}} }

\renewcommand{\gitMarkPref}{{\Large\git git}}

% Esto sirve para poner ecuaciones
\usepackage{wrapfig}
\usepackage{circuitikz}
\usepackage{mathtools}
\usepackage{amssymb}
\usepackage{cancel}
\allowdisplaybreaks

% Esto sirve para poner imágenes{{{
\usepackage{graphicx}
\usepackage{svg}
\usepackage{subcaption}

\usepackage{float}
\usepackage{pgfplots}

\pgfplotsset{compat=1.16}
\graphicspath{ {ima/} }
%}}}
% Colores de los links {{{
\definecolor{red}{HTML}{F22C40}
\definecolor{green}{HTML}{5AB738}
\definecolor{yellow}{HTML}{D5911A}
\definecolor{blue}{HTML}{407EE7}
\definecolor{magenta}{HTML}{6666EA}
\definecolor{cyan}{HTML}{00AD9C}
\definecolor{arch}{HTML}{1793D1}
\definecolor{git}{HTML}{F54D27}

\hypersetup{
	colorlinks=true,
	linkcolor=blue,
	urlcolor=cyan,
	citecolor=magenta,
}
%}}}
% Esto controla a la cabecera {{{
\usepackage{fancyhdr}

\pagestyle{fancy}
\fancyhf{}
\renewcommand{\headrulewidth}{0pt}
\chead{ \textbf{\normalsize{Física II} }}
%\fancyhf[HL]{\includesvg[height=0.8\headheight]{Utec.svg}}
\fancyhf[FL]{\textbf{\thepage}}
\setlength{\headheight}{20pt}
\setlength{\textheight}{675pt}
%}}}
% Título {{{
\title{\textbf{Tarea 3 de física II}}
% Aqui hay que poner a los autores
\author{
		Alberto Oporto Ames\\
		\texttt{alberto.oporto@utec.edu.pe}\\
		%\and <++>\\
		%\texttt{<++>}\\
		}
%}}}
%}}}
% Aquí empieza el documento{{{
\begin{document}
%\maketitle
\thispagestyle{fancy}

\textbf{Alberto Oporto Ames \#139}

% Preguntas {{{
\section*{Preguntas}%

\subsection*{¿Qué es un dielétrico?
¿Cómo y por qué se utiliza en los capacitores?}%

\subsection*{¿De qué depende la rapidez de carga o descarga de un capacitor?}%

% }}}
% Problemas{{{
\section*{Problemas}%
\subsection*{En el siguiente circuito: $C_1 = 1\mu F$,
	$C_2 = 2\mu F$,
	$C_3 = 3\mu F$,
	$C_2 = 4\mu F$,
	y $V_{ad}=15V$. Determina:}%
	%\noindent
	%\begin{tabular}{p{0.5\linewidth}c}
		%\parbox{\linewidth}
		%{
		%\begin{enumerate}[label=\alph*.]
				%\item La carga en el capacitor $C_4$.
				%\item La diferencia de potencial $V_{ab}$.
				%\item La carga en el capacitor $C_1$.
				%\item La energía almacenada en el capacitor $C_2$.
			%\end{enumerate}
		%}&
		%\parbox{0.44\linewidth}
		%{
	%% Circuito {{{
			%\begin{circuitikz}
			%\draw (0,0)
				%node[circ]{} node[above]{a}
				%--
				%(2,0)
				%--
				%(2,1)
				%to [C=$C_1$]
				%(3,1)
				%to [C=$C_2$]
				%(4,1)
				%--
				%(4,0)
				%;
			%\draw (2,0)
				%--
				%(2,-1)
				%to [C=$C_3$]
				%(4,-1)
				%--
				%(4,0)
				%;
			%\draw (4,0)
				%--
				%(6,0)
				%--
				%(6,-1.5)
				%node[circ]{} node[right]{d}
				%--
				%(6,-3)
				%to [C=$C_4$]
				%(0,-3)
				%node[circ]{} node[above]{b}
				%;
		%\end{circuitikz}
	%% }}}
		%}
	%\end{tabular}\\
	%\begin{wrapfigure}{r}{0.42\textwidth}
	%% Circuito {{{
			%\begin{circuitikz}
			%\draw (0,0)
				%node[circ]{} node[above]{a}
				%--
				%(2,0)
				%--
				%(2,1)
				%to [C=$C_1$]
				%(3,1)
				%to [C=$C_2$]
				%(4,1)
				%--
				%(4,0)
				%;
			%\draw (2,0)
				%--
				%(2,-1)
				%to [C=$C_3$]
				%(4,-1)
				%--
				%(4,0)
				%;
			%\draw (4,0)
				%--
				%(6,0)
				%--
				%(6,-1.5)
				%node[circ]{} node[right]{d}
				%--
				%(6,-3)
				%to [C=$C_4$]
				%(0,-3)
				%node[circ]{} node[above]{b}
				%;
		%\end{circuitikz}
	%% }}}
	%\end{wrapfigure}

	%\lipsum[1-2]
	\begin{minipage}{0.5\textwidth}
		\begin{enumerate}[label=\alph*.]
			\item La carga en el capacitor $C_4$.
			\item La diferencia de potencial $V_{ab}$.
			\item La carga en el capacitor $C_1$.
			\item La energía almacenada en el capacitor $C_2$.
		\end{enumerate}
	\end{minipage}
	\begin{minipage}{0.46\textwidth}
		\hfill
	% Circuito {{{
			\begin{circuitikz}
			\draw (0,0)
				node[circ]{} node[above]{a}
				--
				(2,0)
				--
				(2,1)
				to [C=$C_1$]
				(3,1)
				to [C=$C_2$]
				(4,1)
				--
				(4,0)
				;
			\draw (2,0)
				--
				(2,-1)
				to [C=$C_3$]
				(4,-1)
				--
				(4,0)
				;
			\draw (4,0)
				--
				(6,0)
				--
				(6,-1.5)
				node[circ]{} node[right]{d}
				--
				(6,-3)
				to [C=$C_4$]
				(0,-3)
				node[circ]{} node[above]{b}
				;
		\end{circuitikz}
	% }}}
	\end{minipage}
%}}}

\end{document}
%}}}
