\documentclass[10pt, twoside]{article}
% Opciones {{{
\usepackage[pdfa, pdfusetitle, unicode=true]{hyperref}
\usepackage[spanish]{babel}
\usepackage[margin=2.5cm, a4paper]{geometry}
\usepackage{luacode}
\usepackage[shortlabels]{enumitem}
\usepackage{import}
\usepackage{xcolor}
\usepackage{fontspec}
\usepackage[mark, raisemark=0.02\paperheight, marknotags]{gitinfo2}
\usepackage{setspace}
\usepackage{lipsum}

\doublespacing

% Btw I use arch
\setmonofont{InconsolataGo Nerd Font}

\newcommand{\btw}{{\color{arch}\texttt{}} }
\newcommand{\git}{{\color{git}\texttt{}} }

\renewcommand{\gitMarkPref}{{\Large\git git}}

% Esto sirve para poner ecuaciones
\usepackage{wrapfig}
\usepackage{circuitikz}
\usepackage{mathtools}
\usepackage{amssymb}
\usepackage{cancel}

% Sin esto los "<" ">" no funcionan en tikz
\usetikzlibrary{babel}
\allowdisplaybreaks

% Esto sirve para poner imágenes{{{
\usepackage{graphicx}
\usepackage{svg}
\usepackage{subcaption}

\usepackage{float}
\usepackage{pgfplots}

\pgfplotsset{compat=1.16}
\graphicspath{ {ima/} }
%}}}
% Colores de los links {{{
\definecolor{red}{HTML}{F22C40}
\definecolor{green}{HTML}{5AB738}
\definecolor{yellow}{HTML}{D5911A}
\definecolor{blue}{HTML}{407EE7}
\definecolor{magenta}{HTML}{6666EA}
\definecolor{cyan}{HTML}{00AD9C}
\definecolor{arch}{HTML}{1793D1}
\definecolor{git}{HTML}{F54D27}

\hypersetup{
	colorlinks=true,
	linkcolor=blue,
	urlcolor=cyan,
	citecolor=magenta,
}
%}}}
% Esto controla a la cabecera {{{
\usepackage{fancyhdr}

\pagestyle{fancy}
\fancyhf{}
\renewcommand{\headrulewidth}{0pt}
\chead{ \textbf{\normalsize{Física II} }}
%\fancyhf[HL]{\includesvg[height=0.8\headheight]{Utec.svg}}
\fancyhf[FL]{\textbf{\thepage}}
\setlength{\headheight}{20pt}
\setlength{\textheight}{675pt}
%}}}
% Título {{{
\title{\textbf{Tarea 3 de física II}}
% Aqui hay que poner a los autores
\author{
		Alberto Oporto Ames\\
		\texttt{alberto.oporto@utec.edu.pe}\\
		%\and <++>\\
		%\texttt{<++>}\\
		}
%}}}
%}}}
% Aquí empieza el documento{{{
\begin{document}
%\maketitle
\thispagestyle{fancy}

\textbf{Alberto Oporto Ames \#139}

% Preguntas {{{
\section*{Preguntas}%

\subsection*{¿Qué es un dielétrico?
¿Cómo y por qué se utiliza en los capacitores?}%
Algo aislante que está dentro de un capacitor.
Se pone entre las placas de un capacitor para aumentar su capacitancia.

\subsection*{¿De qué depende la rapidez de carga o descarga de un capacitor?}%
Depende de la diferencia de potencial, su capacitancia y la resistencia del circuito.

% }}}
% Problemas{{{
\section*{Problemas}%
\subsection*{En el siguiente circuito: $C_1 = 1\mu F$,
	$C_2 = 2\mu F$,
	$C_3 = 3\mu F$,
	$C_4 = 4\mu F$,
	y $V_{ad}=15V$. Determina:}%
	% Cosas antiguas {{{
	%\noindent
	%\begin{tabular}{p{0.5\linewidth}c}
		%\parbox{\linewidth}
		%{
		%\begin{enumerate}[label=\alph*.]
				%\item La carga en el capacitor $C_4$.
				%\item La diferencia de potencial $V_{ab}$.
				%\item La carga en el capacitor $C_1$.
				%\item La energía almacenada en el capacitor $C_2$.
			%\end{enumerate}
		%}&
		%\parbox{0.44\linewidth}
		%{
	%% Circuito {{{
			%\begin{circuitikz}
			%\draw (0,0)
				%node[circ]{} node[above]{a}
				%--
				%(2,0)
				%--
				%(2,1)
				%to [C=$C_1$]
				%(3,1)
				%to [C=$C_2$]
				%(4,1)
				%--
				%(4,0)
				%;
			%\draw (2,0)
				%--
				%(2,-1)
				%to [C=$C_3$]
				%(4,-1)
				%--
				%(4,0)
				%;
			%\draw (4,0)
				%--
				%(6,0)
				%--
				%(6,-1.5)
				%node[circ]{} node[right]{d}
				%--
				%(6,-3)
				%to [C=$C_4$]
				%(0,-3)
				%node[circ]{} node[above]{b}
				%;
		%\end{circuitikz}
	%% }}}
		%}
	%\end{tabular}\\
	%\begin{wrapfigure}{r}{0.42\textwidth}
	%% Circuito {{{
			%\begin{circuitikz}
			%\draw (0,0)
				%node[circ]{} node[above]{a}
				%--
				%(2,0)
				%--
				%(2,1)
				%to [C=$C_1$]
				%(3,1)
				%to [C=$C_2$]
				%(4,1)
				%--
				%(4,0)
				%;
			%\draw (2,0)
				%--
				%(2,-1)
				%to [C=$C_3$]
				%(4,-1)
				%--
				%(4,0)
				%;
			%\draw (4,0)
				%--
				%(6,0)
				%--
				%(6,-1.5)
				%node[circ]{} node[right]{d}
				%--
				%(6,-3)
				%to [C=$C_4$]
				%(0,-3)
				%node[circ]{} node[above]{b}
				%;
		%\end{circuitikz}
	%% }}}
	%\end{wrapfigure}

	%\lipsum[1-2]
	% }}}
	\begin{minipage}[m]{0.5\textwidth}
		\begin{enumerate}[label=\alph*.]
			\item La carga en el capacitor $C_4$.
			\item La diferencia de potencial $V_{ab}$.
			\item La carga en el capacitor $C_1$.
			\item La energía almacenada en el capacitor $C_2$.
		\end{enumerate}
	\end{minipage}
	\begin{minipage}[m]{0.46\textwidth}
		\hfill
	% Circuito {{{
			\begin{circuitikz}
			\draw (0,0)
				node[circ]{} node[above]{a} -- (2,0)
				-- (2,1)
				to [C=$C_1$] (3,1)
				to [C=$C_2$] (4,1)
				-- (4,0)
				;
			\draw (2,0)
				-- (2,-1)
				to [C=$C_3$] (4,-1)
				-- (4,0)
				;
			\draw (4,0)
				-- (6,0)
				-- (6,-1.5)
				node[circ]{} node[right]{d} -- (6,-3)
				to [C=$C_4$] (0,-3) node[circ]{} node[above]{b}
				;
		\end{circuitikz}
	% }}}
	\end{minipage}
	\begin{enumerate}[label=\alph*.]
		\item
			\begin{align*}
				V_{ad} &= 15V\\
				\\
				C_{123} &= C_3+\Big{(} \frac{1}{C_1}+ \frac{1}{C_2}  \Big{)}^{-1}\\
				C_{123} &= 3\mu F+\Big{(} \frac{1}{1\mu F}+ \frac{1}{2\mu F}  \Big{)}^{-1}\\
				C_{123} &= 3\mu F+ \frac{2}{3} \mu F\\
				C_{123} &= \frac{11}{3} \mu F\\
				\\
				Q_{123} &= C_{123}*V_{ad}\\
				Q_{123} &=
				\boxed
				{
					55\mu C = Q_4
				}
			\end{align*}
		\item
			\begin{align*}
				C_{1234} &= \Big{(} \frac{1}{C_{123}}+ \frac{1}{C_4}  \Big{)}^{-1}\\
				C_{1234} &= \Big{(} \frac{3}{11\mu F}+ \frac{1}{4\mu F}  \Big{)}^{-1}\\
				C_{1234} &= \frac{44}{23} \mu F\\
				\\
				V_{ab} &= \frac{Q_{1234}}{C_{1234}}\\
				V_{ab} &= \frac{55\cancel{\mu}C*23}{44\cancel{\mu}F}\\
				\Aboxed
				{
					V_{ab} &= 28.75V
				}
			\end{align*}
		\item
			\begin{align*}
				V_{123} &= 15V = V_{12}\\
				\\
				Q_{12} &= \frac{C_{12}}{V_{12}} \\
				Q_{12} &= \frac{ 2\mu F }{3*15V} = Q_1\\
				\Aboxed
				{
					Q_1 &= \frac{2}{45} \mu C
				}
			\end{align*}
		\item
			\begin{align*}
					Q_2 &= \frac{2}{45} \mu C\\
					\\
					V_2 &= \frac{\cancel{2\mu} C}{45*\cancel{2\mu} F}\\
					V_2 &= \frac{1}{45}V \\
					\\
					U_2 &= \frac{1}{\cancel{2}} \cancel{2}\mu F*\Big{(} \frac{1}{45}V \Big{)}^2\\
					\Aboxed
					{
						U_2 &= \frac{1}{2025}\mu CV
					}
			\end{align*}
	\end{enumerate}

\subsection*{En el siguiente circuito: $C_1 =C_3=3\mu F$;
	$C_2=C_5=2\mu F$,
	$C_4=4\mu F$,
	$V_{eq}=10V$.
	Todos los interruptores mostrados se encuentran inician abiertos.
	Si se cierra solo el interruptor $S_1$, determina:}%
	\begin{minipage}{0.5\textwidth}
		\begin{enumerate}[label=\alph*.]
			\item La capacitancia equivalente, en $\mu F$.
			\item El voltaje en $C_1$, en V.
		\end{enumerate}
		Si se cierran los interruptores $S_1$ y $S_3$ , determina:
		\begin{enumerate}[label=\alph*.]
			\setcounter{enumi}{2}
			\item El voltaje en $C_1$, en V.
			\item La capacitancia equivalente, en $\mu F$.
		\end{enumerate}
	\end{minipage}
	\begin{minipage}{0.46\textwidth}
		\hfill
	% Circuito {{{
	\begin{circuitikz}[american voltages]
		\draw (0,2)
			-- (0,-1)
			-- (2,-1)
			to [battery, a=$V$, v^>=$ $] (4,-1)
			to [nos, invert, *-*, a=$S_1$] (4.5,-1)
			-- (6,-1)
			to [pC, a=$C_5$] (6,2)
			-- (5,2)
			;
		\draw (5,2)
			-- (5,1)
			to [pC, a^=$C_4$] (3,1)
			to [pC, a^=$C_3$] (1,1)
			-- (1,2)
			-- (0,2)
			;
		\draw (5,2)
			-- (5,3)
			to [pC, a=$C_2$] (3,3)
			to [pC, a=$C_1$] (1,3)
			-- (1,2)
			;
		\draw (3,1)
			-- (3,1.75)
			to [nos, mirror, *-*, a=$S_3$] (3,2.25)
			-- (3,3)
			;
		% Esto es para tener el texto alineado
		\ctikzset{label/align = straight}
		\draw (3,1)
			-- +(135:{sqrt(2)-0.25})
			to [nos, *-*, a=$S_2$] +(135:1.7)
			-- +(135:{sqrt(2)-0.25})
			;
		\end{circuitikz}

	% }}}
	\end{minipage}
	\begin{enumerate}[label=\alph*.]
		\item
			\begin{align*}
				C_{eq} &= \Big{(} \frac{1}{C_{1234}}+ \frac{1}{C_5}  \Big{)}^{-1}\\
				\\
				C_{1234} &= C_{12}+C_{34}\\
				\\
				C_{12} &= \Big{(} \frac{1}{3\mu F} + \frac{1}{2\mu F} \Big{)}^{-1} = \frac{6}{5}\mu F\\
				C_{34} &= \Big{(} \frac{1}{3\mu F} + \frac{1}{4\mu F} \Big{)}^{-1} = \frac{12}{7}\mu F\\
				\\
				C_{1234} &= \frac{6}{5} \mu F + \frac{12}{7} \mu F = \frac{102}{35} \mu F\\
				\\
				C_{eq} &= \Big{(} \frac{35}{102\mu F}+ \frac{1}{2\mu F}  \Big{)}^{-1}\\
				\Aboxed
				{
					C_{eq} &= \frac{51}{43} \mu F
				}
			\end{align*}
		\item
			\begin{align*}
				Q_{1234} &= C_{eq}*V_{eq} = \frac{510}{43} \mu C\\
				\\
				V_{12} &= \frac{Q_{1234}}{C_{1234}}
				= \frac{35*510\cancel{\mu} C}{43*102\cancel{\mu} F}
				=\frac{175}{43}V\\
				\\
				Q_1 &= C_{12}*V_{12}
				= \frac{6}{5} \mu F* \frac{175}{43} V
				= \frac{210}{43} \mu C\\
				\\
				V_1 &= \frac{Q_1}{C_1} \\
				V_1 &= \frac{210\cancel{\mu} C}{43*3\cancel{\mu} F}\\
				\Aboxed
				{
					V_1 &= \frac{70}{43} V
				}
			\end{align*}
	\end{enumerate}
	\noindent\makebox[\linewidth]{\rule{\paperwidth}{0.4pt}}
	\begin{enumerate}[label=\alph*.]
		\setcounter{enumi}{2}
		\item
			\begin{align*}
				C_{eq} &= \Big{(} \frac{1}{C_{1234}}+ \frac{1}{C_5}  \Big{)}^{-1}\\
			\end{align*}
	\end{enumerate}

\subsection*{Se conforma un capacitor de placas planas con múltiples
	distribuciones, como se muestra en la figura.
	Entre los bornes $a$ y $b$ se conecta una batería de $12V$.
	Las constantes de dieléctrico de la mica, polietileno, y baquelita on de $5.4$,
	$2.2$ y $5$ respectivamente. Determina:
	}
\begin{enumerate}
	\item La capacitancia equivalente
	\item La energía almacenada en el capacitor.
	\item La magnitud del campo eléctrico dentro de cada dieléctrico.
\end{enumerate}
\begin{figure}[H]
	\centering
	\includesvg[width=0.8\linewidth]{drawing-1}
\end{figure}
%}}}

\end{document}
%}}}
